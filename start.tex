\documentclass[conference]{IEEEtran}
\IEEEoverridecommandlockouts
% The preceding line is only needed to identify funding in the first footnote. If that is unneeded, please comment it out.
\usepackage{cite}
\usepackage{amsmath,amssymb,amsfonts}
\usepackage{algorithmic}
\usepackage{graphicx}
\usepackage{textcomp}
\usepackage{xcolor}
\usepackage{verbatim}
\usepackage{amsmath,amsfonts,mathtools,bm}
\usepackage{geometry}
\usepackage{subfig}
\usepackage{amsthm}
\usepackage{amssymb}
\usepackage{algorithm}
\usepackage{algorithmic}
\usepackage{newclude}
% \usepackage{appendix}
\usepackage{caption}
% \usepackage{subcaption}
\usepackage{gensymb}
\usepackage{caption}
\usepackage{hyperref}
\usepackage[switch]{lineno}  
\usepackage{cleveref}
\captionsetup[figure]{font=small}


\DeclareMathOperator*{\argmax}{arg\,max}
\DeclareMathOperator*{\argmin}{arg\,min}
\newtheorem{theorem}{Theorem}
\newtheorem*{theorem*}{Theorem}
\newtheorem{lemma}{Lemma}
\newtheorem*{lemma*}{Lemma} 
\newtheorem{claim}{Claim}
\newtheorem{remark}{Remark}
\newenvironment{claimproof}{\noindent\emph{Proof of claim.}}{\hfill$\qed$}

\newcommand{\vect}[1]{\ensuremath{\bm{#1}}}
\newcommand{\E}{\ensuremath{\mathbb{E}}}
\newcommand{\norm}[1]{\left\lVert#1\right\rVert}

%%comment in algorithms
\renewcommand{\algorithmiccomment}[1]{\hfill \# #1}

\newcommand{\denis}[1]{\textcolor{green}{{\footnotesize
#1}}\marginpar{\raggedright\tiny \textcolor{green}{Denis}}}

\newcommand{\thang}[1]{\textcolor{blue}{{\footnotesize
#1}}\marginpar{\raggedright\tiny \textcolor{blue}{Thang}}}

\newcommand{\angan}[1]{\textcolor{blue}{{\footnotesize
#1}}\marginpar{\raggedright\tiny \textcolor{yellow}{Angan}}}

\newcommand{\tuan}[1]{\textcolor{red}{{\footnotesize
#1}}\marginpar{\raggedright\tiny \textcolor{red}{Tuan}}}

\newcommand{\paul}[1]{\textcolor{violet}{{\footnotesize
#1}}\marginpar{\raggedright\tiny \textcolor{violet}{Paul}}}

% \modulolinenumbers[5]


% \def\BibTeX{{\rm B\kern-.05em{\sc i\kern-.025em b}\kern-.08em
%     T\kern-.1667em\lower.7ex\hbox{E}\kern-.125emX}}
\begin{document}

\title{Decentralized Meta Frank Wolfe for Online Neural Network Optimization}

% \author{
% \IEEEauthorblockN{Angan Mitra}
% \IEEEauthorblockA{\textit{Qarnot Computing} \\
% % \textit{name of organization (of Aff.)}\\
% Paris, France \\
% angan.mitra@qarnot-computing.com}
% \and\IEEEauthorblockN{Kim Thang Nguyen}
% \IEEEauthorblockA{University Paris Saclay, France\\
% % \textit{name of organization (of Aff.)}\\
% Paris, France \\
% kimthang.nguyen@univ-evry.fr}
% \and
% \IEEEauthorblockN{ Tuan-Anh Nguyen, \\ Denis Trystram, Paul Youssef}
% \IEEEauthorblockA{\textit{Laboratory of Informatics} \\
% % \textit{name of organization (of Aff.)}\\
% Grenoble, France \\
% firstname.lastname@imag.fr}

% }

\maketitle


\begin{abstract}
The design of decentralized learning algorithms is important in the fast-growing world in which data are distributed over participants with limited local computation resources and  communication. In this direction, we propose an online algorithm minimizing non-convex loss functions aggregated from individual data/models distributed over a network. We provide the theoretical performance guarantee of our algorithm and demonstrate its utility on a real life smart building.   

\begin{comment}
Streaming data from IoT devices usually needs a data-lake or a database to dump sensor values if not processed on the go. 
A report in 2019 by Cisco predicts the data storage capacity to reach around 50 Zeta Bytes by 2030. 
In parallel, generated data can be too sensitive to share depending on it's source for example images from a video-surveillance room. 
Recent trends in artificial intelligence like federated learning talks about privacy by design algorithms where machine-learnt functions are shared instead of raw data.
Such literature till now mainly focuses on centralized yet distributed systems where a mediator node controls the process of federation. 
Such a methodology always needs to trust an external third party, so that the latter does not share insights from participating models to the outside world.
Decentralization of learning techniques has the potential to relieve the dependency of such mediators through peer to peer (P2P) knowledge exchanges.
% Past work has not focused much on linkages between nodes in a knowledge sharing environment.
Our algorithms are designed to update deep learning models in an online manner for every node in a network topology of connected learners. 
We demonstrate the utility of decentralized learning on a real life smart building where zones across multiple floors are considered as a connected set of learners. 
Each zone has a neural network that is trained in an online manner for predictive forecasting of indoor temperature.
The work gives an experimental insight to understand how certain topologies of orchestration of learners affect the learning.
Finally we highlight the role of decentralization by bench-marking against the state of the art Meta Frank Wolfe implementation with exact gradients with an experimental approximation ratio of 1.4.\end{comment}



\end{abstract}

\begin{IEEEkeywords}
Decentralized Online Learning, Non Convex Functions, Neural Networks,  Smart building application.
\end{IEEEkeywords}


\include*{chapters/introduction}
\include*{chapters/relatedwork}
\include*{chapters/formulation}
\include*{chapters/experiments}
\include*{chapters/conclusion}


% \section*{Acknowledgment}
% Project 

\bibliographystyle{IEEEtran}
\bibliography{ref.bib}

\onecolumn 
\appendix

\subsection{Technical details in Section \ref{sec:exact}} 

Recall that $\overline{\vect{x}}^{t}_{\ell} := \frac{1}{n} \sum_{j=1}^{n} \vect{x}^{t}_{j,\ell}$.
First, we show some property of $\overline{\vect{x}}^{t}_{\ell}$'s. 

\begin{lemma}\label{lmm:avg}
For every $1 \leq t \leq T$ and $1 \leq \ell \leq L$, it holds that 
\begin{equation}
  \overline{\vect{x}}^t_{\ell+1} - \overline{\vect{x}}^t_{\ell} = \eta_{\ell} \left( \frac{1}{n}\sum_{i=1}^{n} \vect{v}^t_{i,\ell} - \overline{\vect{x}}^t_{\ell}\right)
\end{equation}
\end{lemma} 

\begin{proof}
\begin{linenomath}
\begin{align*}
\overline{\vect{x}}_{\ell+1}^t 
&= \frac{1}{n} \sum_{i=1}^{n} \vect{x}^{t}_{i,\ell + 1} \tag{Definition of $\overline{\vect{x}}_{\ell+1}^t$}
\\
&= \frac{1}{n} \sum_{i=1}^{n} \left ((1-\eta_{\ell}) \vect{y}^{t}_{i,\ell} + \eta_{\ell} \vect{v}_{i,\ell}^t \right )  \tag{Definition of $\vect{x}_{i,\ell}^t$} 
\\
&= \frac{1}{n} \sum_{i=1}^{n} \left [ (1-\eta_{\ell}) \left ( \sum_{j=1}^n \vect{W}_{ij} \vect{x}_{j,\ell}^t  \right ) + \eta_{\ell} \vect{v}_{i,\ell}^t \right ] \tag{Definition of $\vect{y}_{i,\ell}^t$}
\\
&= (1 - \eta_{\ell}) \frac{1}{n} \sum_{i=1}^n \left [ \sum_{j=1}^n \vect{W}_{ij}\vect{x}_{j,\ell}^t \right ] + \frac{1}{n}\eta_{\ell} \sum_{i=1}^n \vect{v}_{i,\ell}^t
\\
&= (1 - \eta_{\ell}) \frac{1}{n} \sum_{j=1}^n \left [ \vect{x}_{j,\ell}^t \sum_{i=1}^n \vect{W}_{ij} \right ] + \frac{1}{n} \eta_{\ell} \sum_{i=1}^n \vect{v}_{i,\ell}^t
\\
&= (1 - \eta_{\ell}) \frac{1}{n} \sum_{j=1}^n \vect{x}_{j,\ell}^t + \frac{1}{n} \eta_{\ell} \sum_{i=1}^n \vect{v}_{i,\ell}^t \tag{$\sum_{i=1}^n W_{ij} =1$ for every $j$}
\\
&= (1 - \eta_{\ell}) \overline{\vect{x}}_{\ell}^t + \frac{1}{n} \eta_{\ell} \sum_{i=1}^n \vect{v}_{i,\ell}^t
\\
&= \overline{\vect{x}}_{\ell}^t + \eta_{\ell} \left ( \frac{1}{n} \sum_{i=1}^n \vect{v}_{i,\ell}^t - \overline{\vect{x}}_{\ell}^t \right )
\end{align*}
\end{linenomath}
where we use a property of $W$ which is $\sum_{i = 1}^{n} W_{ij} = 1$ for every $j$.
\end{proof}


\setcounter{lemma}{1}

\begin{lemma}
For every $1 \leq i \leq n$ and $1 \leq \ell \leq L$, it holds that
\begin{align*}
    \max_{\vect{o} \in \mathcal{K}}\langle \nabla F^{t} (\vect{x}^t_{i,\ell}), \vect{x}^t_{i,\ell} - \vect{o} \rangle 
     \leq \max_{\vect{o} \in \mathcal{K}}\langle \nabla F^{t} (\overline{\vect{x}}^t_{\ell}), \overline{\vect{x}}^t_{\ell} - \vect{o} \rangle
    + \left(\beta D + G \right)C_p \frac{\log L}{L}.
\end{align*}
\end{lemma}
%
\begin{proof} 
Fix $1 \leq i \leq n$ and $1 \leq \ell \leq L$. We have
    \begin{align*}
        \langle \nabla F^{t} (\vect{x}^t_{i,\ell}), \vect{x}^t_{i,\ell} - \vect{o} \rangle 
        & = \langle \nabla F^{t} (\overline{\vect{x}}^t_{\ell}), \overline{\vect{x}}^t_{\ell} - \vect{o} \rangle + \langle \nabla F^{t} (\vect{x}^t_{i,\ell}) - \nabla F^{t} (\overline{\vect{x}}^t_\ell), \overline{\vect{x}}^t_\ell - \vect{o} \rangle + \langle \nabla F^{t} (\vect{x}^t_{i,\ell}), \vect{x}^t_{i,\ell} - \overline{\vect{x}}^t_\ell \rangle
    \end{align*}
Therefore,
    \begin{align*}
        \max_{\vect{o} \in \mathcal{K}}\langle \nabla F^{t} (\vect{x}^t_{i,\ell}), \vect{x}^t_{i,\ell} - \vect{o} \rangle 
        & \leq \max_{\vect{o} \in \mathcal{K}}\langle \nabla F^{t} (\overline{\vect{x}}^t_\ell), \overline{\vect{x}}^t_\ell - \vect{o} \rangle + \max_{\vect{o} \in \mathcal{K}} \langle \nabla F (\vect{x}^t_{i,\ell}) - \nabla F^{t} (\overline{\vect{x}}^t_\ell), \overline{\vect{x}}^t_\ell - \vect{o} \rangle \\
        	& \qquad + \langle \nabla F^{t} (\vect{x}^t_{i,\ell}), \vect{x}^t_{i,\ell} - \overline{\vect{x}}^t_\ell \rangle \\
        & \leq \max_{\vect{o} \in \mathcal{K}}\langle \nabla F^{t} (\overline{\vect{x}}^t_\ell), \overline{\vect{x}}^t_\ell - \vect{o} \rangle  + (\beta D + G) \E{\|\vect{x}^t_{i,\ell} - \overline{\vect{x}}^t_\ell\|} \\
        & \leq \max_{\vect{o} \in \mathcal{K}}\langle \nabla F^{t} (\overline{\vect{x}}^t_\ell), \overline{\vect{x}}^t_\ell - \vect{o} \rangle + (\beta D + G)C_p \frac{\log L}{L}.
    \end{align*}
\end{proof}

\setcounter{theorem}{0}

\begin{theorem}
%\label{thm:gap}
Let $\mathcal{K}$ be a convex set with diameter D. Assume that functions $F^{t}$ (possibly non convex) are $\beta$-smooth and G-Lipschitz for every t. With the choice of step size $\eta_{\ell} = \min\left(1, \frac{A}{\ell^{\alpha}}\right)$ where $A \in \mathbb{R_{+}}$ and $\alpha \in (0,1)$. 
Then, Algorithm 1 guarantees that for all $1 \leq i \leq n$:
%
    \begin{align*}
       \max_{\vect{o} \in \mathcal{K}}\frac{1}{T} \sum_{t=1}^{T} \E_{\vect{x}^t_i} \bigl [\langle \nabla F^{t}(\vect{x}^t_{i}), \vect{x}^t_{i} - \vect{o}\rangle \bigr] \nonumber 
       & \leq \frac{GDA^{-1}}{L^{1-\alpha}}  
         + \frac{AD^2 \beta/2}{L^{\alpha}(1-\alpha)} + O \left(\mathcal{R}^{T}\right) \notag \\
        & \quad + \left(\left( \beta C_p + C_d \right)D + \left(\beta D + G \right)C_p \right)\frac{\log L}{L}
    \end{align*}
where $\mathcal{R}^T$ is the regret of online linear minimization oracles.
Choosing $L=T$, $\alpha = 1/2$ and oracles as gradient descent or follow-the-perturbed-leader with regret $\mathcal{R}^T =
O\bigl(T^{-1/2}\bigr)$, we obtain the gap convergence rate of $O\bigl(T^{-1/2}\bigr)$.
\end{theorem}
%
\begin{proof}
We complete the proof presented in the main text by giving the details of Inequality (\ref{tk:connection}) and showing that from Inequality~(\ref{tk:gap_ell}), i.e., 
%We complete the proof presented in the main text by showing that from Inequality~(\ref{tk:gap_ell}), i.e., 

    \begin{align}
        \mathcal{G}^t_{\ell} 
        & \leq \frac{L^{\alpha}}{A} \left( F^{t} \left( \overline{\vect{x}}^t_{\ell} \right) - F^{t} \left( \overline{\vect{x}}^t_{\ell+1} \right) \right) 
        + \frac{(\beta C_p + C_d)D}{\ell} 
        + \frac{1}{n} \sum_{i=1}^{n}\langle \vect{d}^t_{i,\ell}, \vect{v}^t_{i,\ell}- \vect{o}^t_{\ell} \rangle 
        + \eta_{\ell}D^2 \frac{\beta}{2} 
		\tag{\ref{tk:gap_ell}}
    \end{align}
  %  
 one can deduce that
 %
     \begin{align*}
        \E_{\overline{\vect{x}}^t} \left[ \mathcal{G}^t \right]
        & \leq \frac{GDA^{-1}}{L^{1-\alpha}} 
        + \left( \beta C_p + C_d \right)D\frac{\log L}{L}
        + \frac{1}{nL}\sum_{\ell=1}^{L}\sum_{i=1}^{n}\langle \vect{d}^t_{i,\ell}, \vect{v}^t_{i,\ell} - \vect{o}^t_{\ell} \rangle 
        + \frac{AD^2 \beta/2}{L^{\alpha}(1-\alpha)} \tag{\ref{tk:exp_gap}}
    \end{align*}
%
For all $\alpha \in (0, 1)$, 
%
\begin{equation*}
    \sum_{\ell=1}^{L} \frac{1}{\ell^{\alpha}} \leq 1 + \int_{1}^L \frac{1}{s^{\alpha}}ds = 1 + \frac{L^{1-\alpha} - 1}{1-\alpha} \leq \frac{L^{1-\alpha}}{1-\alpha}
\end{equation*} 
%
By definition of $\eta_{\ell} = \frac{A}{\ell^{\alpha}}$, $G$-Lipschitz property of $F$ and Lemma~\ref{lmm:avg}, from Inequality~(\ref{tk:gap_ell}), we deduce that: 
%
    \begin{align*}
        \E_{\overline{\vect{x}}^t} \bigl[\mathcal{G}^t] = \frac{1}{L}\sum_{\ell=1}^{L} \mathcal{G}^t_{\ell}
        & \leq \frac{L^{\alpha}GDA^{-1}}{L} 
        + \frac{\left(\beta C_p + C_d \right)D}{L}\sum_{\ell=1}^{L} \frac{1}{\ell} 
        + \frac{1}{nL} \sum_{\ell=1}^{L}\sum_{i=1}^{n} \langle \vect{d}^t_{i,\ell}, \vect{v}^t_{i,\ell} - \vect{o}^t_{\ell} \rangle \\
        & \quad + \frac{AD^2 \beta/2}{L} \sum_{\ell=1}^{L} \frac{1}{\ell^{\alpha}} \notag \\
        & \leq  \frac{GDA^{-1}}{L^{1-\alpha}} 
        + \left(\beta C_p + C_d \right)D \frac{\log L}{L}
        + \frac{1}{nL} \sum_{\ell=1}^{L}\sum_{i=1}^{n} \langle \vect{d}^t_{i,\ell}, \vect{v}^t_{i,\ell} - \vect{o}^t_{\ell} \rangle \\  
        & \quad + \frac{AD^2 \beta/2}{L} \frac{L^{1-\alpha}}{1-\alpha} \tag{$\sum_{\ell=1}^{L} \frac{1}{\ell} \leq \log L$} \\
        & \leq \frac{GDA^{-1}}{L^{1-\alpha}} 
        + \left( \beta C_p + C_d \right)D\frac{\log L}{L}
        + \frac{1}{nL}\sum_{\ell=1}^{L}\sum_{i=1}^{n}\langle \vect{d}^t_{i,\ell}, \vect{v}^t_{i,\ell} - \vect{o}^t_{\ell} \rangle \\
        & \quad + \frac{AD^2 \beta/2}{L^{\alpha}(1-\alpha)} 
    \end{align*}
%
For Inequality (\ref{tk:connection}), we use Lemma~\ref{lemma:final_step} with the following analysis
    \begin{align*}	%\label{tk:connection}
    \frac{1}{T}\sum_{t=1}^{T} \E_{\vect{x}^t_i} \bigl [ \max_{\vect{o} \in \mathcal{K}}\langle \nabla F^{t}(\vect{x}^t_{i}), \vect{x}^t_{i} - \vect{o}\rangle \bigr] 
    & \leq \frac{1}{T}\sum_{t=1}^{T} \frac{1}{L} \sum_{\ell=1}^{L} \bigl [ \max_{\vect{o} \in \mathcal{K}}\langle \nabla F^{t}(\vect{x}^t_{i,\ell}), \vect{x}^t_{i,\ell} - \vect{o}\rangle \bigr] \\
    & \leq \frac{1}{T}\sum_{t=1}^{T} \frac{1}{L} \sum_{\ell=1}^{L} \biggl[ \max_{\vect{o} \in \mathcal{K}}
    		\langle \nabla F^{t} (\overline{\vect{x}}^t_{\ell}), \overline{\vect{x}}^t_{\ell} - \vect{o} \rangle \notag 
    +  \left(\beta D + G \right)C_p \frac{\log L}{L}
    		\biggr]  \tag{by Lemma \ref{lemma:final_step}} \\
    &= \frac{1}{T}\sum_{t=1}^{T} \E_{\overline{\vect{x}}^t} \left[\mathcal{G}^t\right] +  \left(\beta D + G \right) C_p \frac{\log L}{L} 	
    \end{align*}
 %
The last equality holds since 
     \begin{align*}
    \E_{\overline{\vect{x}}^t} \left[\mathcal{G}^t\right] = \E_{\overline{\vect{x}}^t} {\left[\max_{\vect{o} \in \mathcal{K}}\langle \nabla F_t (\overline{\vect{x}}^t_{\ell}), \overline{\vect{x}}^t_{\ell} - \vect{o} \rangle \right]}
    \end{align*}
 That completes the proof of the theorem.
\end{proof}


%\begin{proof}
%By $\beta$-smoothness, $\forall \ell \in \{1, \cdots, L\}$: 
%
%\begin{equation}\label{eq:smth}
%    F^{t}\left( \overline{\vect{x}}^{t}_{\ell+1} \right) - F^{t} \left( \overline{\vect{x}}^{t}_{\ell} \right) \leq \langle \nabla F^{t} \left( \overline{\vect{x}}^t_{\ell} \right), \overline{\vect{x}}^t_{\ell+1} - \overline{\vect{x}}^t_{\ell} \rangle + \frac{\beta}{2} \left \| \overline{\vect{x}}^t_{\ell+1} - \overline{\vect{x}}^t_{\ell} \right\|^{2}
%\end{equation} 
%\newline 
%Using Lemma~\ref{lmm:avg}, the inner product in~(\ref{eq:smth}) can be re-written as : 
%\begin{equation}
%\label{eq:inner_beta}
%    \begin{aligned}
%        \left \langle \nabla F^{t} \left( \overline{\vect{x}}^t_{\ell} \right), \overline{\vect{x}}^t_{\ell+1} - \overline{\vect{x}}^t_{\ell} \right \rangle
%        & = \eta_{\ell} \left \langle \nabla F^{t} \left( \overline{\vect{x}}^t_{\ell} \right), \frac{1}{n}\sum_{i=1}^{n} \vect{v}^t_{i,\ell} - \overline{\vect{x}}^t_{\ell} \right \rangle \\
%        &= \eta_{\ell} \left \langle \nabla F^{t} \left( \overline{\vect{x}}^t_{\ell} \right),  \frac{1}{n} \biggl(\sum_{i=1}^{n} \vect{v}^t_{i,\ell} - n \cdot \overline{\vect{x}}^t_{\ell} \biggr) \right \rangle \\
%        &=  \frac{\eta_{\ell}}{n}\sum_{i=1}^{n} \left \langle \nabla F^{t} \left( \overline{\vect{x}}^t_{\ell} \right), \vect{v}^t_{i,\ell} - \overline{\vect{x}}^t_{\ell} \right \rangle
%    \end{aligned}
%\end{equation} 
%\newline
%Define $\vect{a}^t_{\ell}$ as:
%\begin{linenomath}
%    \begin{align*}
%        \vect{a}^t_{\ell} = \argmin_{\vect{a}^{\prime} \in \mathcal{K}}\langle \nabla F(\overline{\vect{x}}^t_{\ell}),\vect{a}^{\prime} \rangle
%    \end{align*}
%\end{linenomath}
%and recall that :
%\begin{equation*}
% \mathcal{G}^t_{\ell} := \max_{\vect{a}^{\prime} \in \mathcal{K}}\langle \nabla F(\overline{\vect{x}}^t_{\ell}), \overline{\vect{x}}^t_{\ell} - \vect{a}^{\prime}\rangle = \langle \nabla F(\overline{\vect{x}}^t_{\ell}), \overline{\vect{x}}^t_{\ell} - \vect{a}^t_{\ell}\rangle
%\end{equation*} 
%We have :
%\begin{linenomath}
%    \begin{align*}
%        \left \langle \nabla F^{t} \left( \overline{\vect{x}}^t_{\ell} \right) , \vect{v}^t_{i,\ell} - \overline{\vect{x}}^t_{\ell} \right \rangle 
%        & =\langle \nabla F^{t} \left( \overline{\vect{x}}^t_{\ell} \right) - \vect{d}^t_{i,\ell}, \vect{v}^t_{i,\ell} - \vect{a}^t_{\ell} \rangle 
%         + \langle \vect{d}^t_{i,\ell}, \vect{v}^t_{i,\ell} - \vect{a}^t_{\ell} \rangle 
%         + \langle \nabla F^{t} \left( \overline{\vect{x}}^t_{\ell} \right), \vect{a}^t_{\ell} - \overline{\vect{x}}^t_{\ell} \rangle \\
%        &\leq \|\nabla F^{t} \left(\overline{\vect{x}}^t_{\ell} \right) - \vect{d}^t_{i, \ell} \| \|\vect{v}^t_{i,\ell} - \vect{a}^t_{\ell}\| + \langle \vect{d}^t_{i,\ell}, \vect{v}^t_{i,\ell} - \vect{a}^t_{\ell} \rangle 
%         + \langle \nabla F^{t} \left( \overline{\vect{x}}^t_{\ell} \right), \vect{a}^t_{\ell} - \overline{\vect{x}}^t_{\ell} \rangle \tag{by Cauchy-Schwarz}\\
%         & \leq \|\nabla F^{t} \left(\overline{\vect{x}}^t_{\ell} \right) - \vect{d}^t_{i, \ell} \|D + \langle \vect{d}^t_{i,\ell}, \vect{v}^t_{i,\ell} - \vect{a}^t_{\ell} \rangle 
%         + \langle \nabla F^{t} \left( \overline{\vect{x}}^t_{\ell} \right), \vect{a}^t_{\ell} - \overline{\vect{x}}^t_{\ell} \rangle 
%    \end{align*}
%\end{linenomath} 
%\newline
%Using Lemma~\ref{lem:convergence} and $\beta$-smoothness of $F^t$,
%\begin{linenomath}
%    \begin{align*}
%        \left\| \nabla F^{t} \left(\overline{\vect{x}}^t_{\ell} \right) - \vect{d}^t_{i, \ell} \right\|
%        & \leq \left\| \nabla F^{t} \left(\overline{\vect{x}}^t_{\ell} \right) - \frac{1}{n}\sum_{i=1}^n \nabla f^t_i (\vect{y}^t_{i,\ell}) \right\| + \left \lVert\frac{1}{n}\sum_{i=1}^n \nabla f^t_i (\vect{y}^t_{i,\ell}) - \vect{d}^t_{i, \ell} \right\lVert \\
%        & \leq \left\| \frac{1}{n}\sum_{i=1}^n \nabla f^{t}_{i} \left(\overline{\vect{x}}^t_{\ell} \right) - \frac{1}{n}\sum_{i=1}^n \nabla f^t_i (\vect{y}^t_{i,\ell}) \right\| + \left \lVert\frac{1}{n}\sum_{i=1}^n \nabla f^t_i (\vect{y}^t_{i,\ell}) - \vect{d}^t_{i, \ell} \right\lVert \\
%        & \leq  \frac{1}{n}\sum_{i=1}^n \left \|\nabla f^{t}_{i} \left(\overline{\vect{x}}^t_{\ell} \right) -  \nabla f^t_i (\vect{y}^t_{i,\ell}) \right\| + \left \lVert\frac{1}{n}\sum_{i=1}^n \nabla f^t_i (\vect{y}^t_{i,\ell}) - \vect{d}^t_{i, \ell} \right\lVert \\
%        & \leq \frac{\beta}{n} \sum_{i=1}^n \left \| \overline{\vect{x}}^t_{\ell} - \vect{y}^t_{i, \ell} \right\| + \left\|\frac{1}{n}\sum_{i=1}^n \nabla f^t_i (\vect{y}^t_{i,\ell}) - \vect{d}^t_{i, \ell} \right\| \tag{by $\beta$ smoothness}\\
%        & \leq \frac{\beta C_p + C_d}{\ell}  \tag{by Lemma \ref{lem:convergence}}
%    \end{align*}
%\end{linenomath}
%Thus,
%\begin{linenomath}
%    \begin{align*}
%        \left \langle \nabla F^{t} \left( \overline{\vect{x}}^t_{\ell} \right) , \vect{v}^t_{i,\ell} - \overline{\vect{x}}^t_{\ell} \right \rangle 
%        & \leq \left( \frac{\beta C_p + C_d}{\ell}\right)D + \langle \vect{d}^t_{i,\ell}, \vect{v}^t_{i,\ell} - \vect{a}^t_{\ell} \rangle - \mathcal{G}^t_{\ell} 
%    \end{align*}
%\end{linenomath}
%Upper bound  the right hand side of~(\ref{eq:inner_beta}) by the above inequality, we have :
%
%
%\begin{equation}
%\label{eq:inner_beta2}
%    \begin{aligned}
%        \left \langle \nabla F^{t} \left( \overline{\vect{x}}^t_{\ell} \right), \overline{\vect{x}}^t_{\ell+1} - \overline{\vect{x}}^t_{\ell} \right \rangle
%        & \leq \eta_{\ell}\left( \frac{\beta C_p + C_d}{\ell}\right)D + \frac{\eta_{\ell}}{n} \sum_{i=1}^{n}\langle \vect{d}^t_{i,\ell}, \vect{v}^t_{i,\ell} - \vect{a}^t_{\ell} \rangle - \eta_{\ell}\mathcal{G}^t_{\ell} 
%    \end{aligned}
%\end{equation}  Combining~(\ref{eq:smth}) with~(\ref{eq:inner_beta2}) and re-arrange the terms, as $\eta_{\ell} = \frac{A}{\ell^{\alpha}}$, we have : 
%\begin{equation}
%\label{eq:eta_gap_ell}
%    \begin{aligned}
%         \eta_{\ell} \mathcal{G}^t_{\ell} 
%        & \leq F^{t} \left( \overline{\vect{x}}^t_{\ell} \right) - F^{t} \left( \overline{\vect{x}}^t_{\ell+1} \right) 
%        + \eta_{\ell}\left( \frac{\beta C_p + C_d}{\ell}\right)D + \frac{\eta_{\ell}}{n} \sum_{i=1}^{n}\langle \vect{d}^t_{i,\ell}, \vect{v}^t_{i,\ell} - \vect{a}^t_{\ell} \rangle + \eta^2_{\ell}D^2 \frac{\beta}{2} \\
%        & \leq F^{t} \left( \overline{\vect{x}}^t_{\ell} \right) - F^{t} \left( \overline{\vect{x}}^t_{\ell+1} \right)
%        + \left[ \left( \beta C_p + C_d \right)AD + A^2 D^2 \beta/2 \right] \frac{1}{\ell^{2\alpha}} + \frac{\eta_{\ell}}{n} \sum_{i=1}^{n}\langle \vect{d}^t_{i,\ell}, \vect{v}^t_{i,\ell} - \vect{a}^t_{\ell} \rangle 
%    \end{aligned}
%\end{equation}  Let $\mathcal{G}^t$ be a random variable such that $\mathcal{G}^t = \mathcal{G}^t_{\ell}$ with probability $\frac{1}{L}$. Recall that $\eta_{L} \leq \eta_{\ell}$ : 
%%\begin{equation}
%%    \begin{aligned}
%%        \eta_{L} \E \left[ \mathcal{G}^t \right] \leq 
%%        \eta_{L}\frac{1}{L}\sum_{\ell=1}^L\mathcal{G}^t_{\ell%} \leq \frac{1}{L}\sum_{\ell=1}^L \eta_{\ell} %\mathcal{G}^t_{\ell}
%%    \end{aligned}
%%\end{equation}
%\begin{equation}
%    \begin{aligned}
%        \eta_{L} \E \left[ \mathcal{G}^t \right] \leq 
%        \frac{1}{L}\sum_{\ell=1}^L\eta_{\ell}\mathcal{G}^t_{\ell}
%        & \leq \frac{1}{L}\sum_{\ell=1}^{L}F^{t} \left( \overline{\vect{x}}^t_{\ell} \right) - F^{t} \left( \overline{\vect{x}}^t_{\ell+1} \right)
%        + \frac{ \left( \beta C_p + C_d \right)AD + A^2 D^2 \beta/2}{L}\sum_{\ell=1}^{L} \frac{1}{\ell^{2\alpha}}\\
%        &+ \frac{1}{L}\sum_{\ell=1}^{L}\frac{\eta_{\ell}}{n} \sum_{i=1}^{n}\langle \vect{d}^t_{i,\ell}, \vect{v}^t_{i,\ell} - \vect{a}^t_{\ell} \rangle 
%    \end{aligned}
%\end{equation} 
%%\thang{line too long}
%\newline
%For all $\alpha \in (0, 1)$, 
%\begin{equation}
%\begin{aligned}
%    \sum_{\ell=1}^{L} \frac{1}{\ell^{\alpha}} \leq 1 + \int_{1}^L \frac{1}{s^{\alpha}} = 1 + \frac{L^{1-\alpha} - 1}{1-\alpha} \leq \frac{L^{1-\alpha}}{1-\alpha}
%\end{aligned}
%\end{equation} In case $\alpha < 0.5$ :
%\begin{equation}
%    \begin{aligned}
%        \sum_{\ell=1}^{L} \frac{1}{\ell^{2\alpha}}
%        & \leq 1 + \int_{1}^{L} \frac{1}{s^{2\alpha}}ds \\
%        & \leq 1 +  \begin{dcases}
%            \frac{L^{1-2\alpha}}{1-2\alpha} - \frac{1}{1-2\alpha} \leq \frac{L^{1-2\alpha} - 2\alpha}{1-2\alpha} \leq \frac{L^{1-2\alpha}}{1-2\alpha} &\text{ if } 0 < \alpha < 0.5\\
%            \frac{1-L^{-(2\alpha-1)}}{2\alpha-1} = \frac{2\alpha - L^{1-2\alpha}}{2\alpha - 1} \leq \frac{2\alpha}{2\alpha - 1} &\text{ if } 0.5 < \alpha < 1 \\
%            \end{dcases} 
%    \end{aligned}
%\end{equation} In case $\alpha > 0.5$ :
%\begin{equation}
%    \begin{aligned}
%        \sum_{\ell=1}^{L} \frac{1}{\ell^{2\alpha}} \leq 1 + \int_{1}^{L} \frac{1}{s^{2\alpha}}ds \leq 1 + \frac{1-L^{-(2\alpha-1)}}{2\alpha-1} = \frac{2\alpha - L^{1-2\alpha}}{2\alpha - 1} \leq \frac{2\alpha}{2\alpha - 1} 
%    \end{aligned}
%\end{equation}
%By definition of $\eta_{L} = \frac{A}{L^{\alpha}}$, G-Lipschitz property of $F$ and bounded convexity of $\mathcal{K}$, we have then,
%\begin{equation}
%    \begin{aligned}
%        \eta_L \E \bigl[\mathcal{G}^t] 
%        %& \leq \frac{GD}{L}
%        %+ \frac{\left(\beta C_p + C_d\right)AD}{L}\frac{L^{1-2\alpha}}{1-2\alpha}
%        % + \frac{1}{nL}\sum_{\ell=1}^{L}\sum_{i=1}^{n}\eta_{\ell}\langle \vect{d}^t_{i,\ell}, \vect{v}^t_{i,\ell} - \vect{a}^t_{\ell} \rangle  + \frac{A^2 D^2 \beta / 2}{L}\frac{L^{1-2\alpha}}{1-2\alpha}\\
%        & \leq \frac{GD}{L} + \frac{1}{nL}\sum_{\ell=1}^{L}\sum_{i=1}^{n}\frac{A}{\ell^\alpha}\langle \vect{d}^t_{i,\ell}, \vect{v}^t_{i,\ell} - \vect{a}^t_{\ell} \rangle \\
%        & + \begin{dcases}
%            \frac{\left(\beta C_p + C_d\right)AD + A^2 D^2 \beta / 2}{L} \frac{L^{1-2\alpha}}{1-2\alpha} &\text{ if } 0 < \alpha < 0.5\\
%            \frac{\left(\beta C_p + C_d\right)AD + A^2 D^2 \beta / 2}{L}\frac{2\alpha}{2\alpha-1} &\text{ if } 0.5 < \alpha < 1 \\
%        \end{dcases}
%    \end{aligned}
%\end{equation} Thus, 
%\begin{equation}
%    \begin{aligned}
%        \E \bigl[\mathcal{G}^t]
%        & \leq \frac{GDA^{-1}}{L^{1-\alpha}} + \frac{1}{nL^{1-\alpha}}\sum_{\ell=1}^{L}\sum_{i=1}^{n}\frac{1}{\ell^{\alpha}}\langle \vect{d}^t_{i,\ell}, \vect{v}^t_{i,\ell} - \vect{a}^t_{\ell} \rangle \\
%        & + \begin{dcases}
%            \frac{\left(\beta C_p + C_d\right)D + A D^2 \beta / 2}{L^{\alpha}}(1-2\alpha)^{-1} &\text{ if } 0 < \alpha < 0.5\\
%            \frac{\left(\beta C_p + C_d\right)D + A D^2 \beta / 2}{L^{1-\alpha}}\frac{2\alpha}{2\alpha-1} &\text{ if } 0.5 < \alpha < 1 \\
%        \end{dcases}
%    \end{aligned}
%\end{equation} Taking the mean on $T$, if the chosen oracle has regret of $O \left(\sqrt{T}\right)$, the second term on the right hand side of the above equation is bounded by $O \left(\frac{(1-\alpha)^{-1}}{\sqrt{T}} \right)$,
%\begin{equation}
%\label{eq:case1_finalbound}
%    \begin{aligned}
%        \frac{1}{T}\sum_{t=1}^{T} \E \bigl[\mathcal{G}^t]  
%        & \leq \frac{GDA^{-1}}{L^{1-\alpha}} + O \left(\frac{(1-\alpha)^{-1}}{\sqrt{T}} \right) 
%        + \begin{dcases}
%            \frac{\left(\beta C_p + C_d\right)D + A D^2 \beta / 2}{L^{\alpha}}(1-2\alpha)^{-1} &\text{ if } 0 < \alpha < 0.5\\
%             \frac{\left(\beta C_p + C_d\right)D + A D^2 \beta / 2}{L^{1-\alpha}}\frac{2\alpha}{2\alpha-1} &\text{ if } 0.5 < \alpha < 1 \\
%        \end{dcases}
%    \end{aligned}
%\end{equation} 
%Recall that,
%\begin{align*}
%    \E \left[\mathcal{G}^t\right] = \E{\left[\max_{a \in \mathcal{K}}\langle \nabla F_t (\overline{\vect{x}}^t), \overline{\vect{x}}^t - a \rangle \right]}
%\end{align*}
%The theorem followed by applying lemma \ref{lemma:final_step} and setting $L=T$ in~(\ref{eq:case1_finalbound}) %and~(\ref{eq:case2_finalbound}).
%
%\begin{comment}
%\textbf{In case $\alpha > 0.5$ :}
%\begin{equation}
%    \begin{aligned}
%        \sum_{\ell=1}^{L} \frac{1}{\ell^{2\alpha}} \leq 1 + \int_{1}^{L} \frac{1}{s^{2\alpha}}ds \leq 1 + \frac{1-L^{-(2\alpha-1)}}{2\alpha-1} = \frac{2\alpha - L^{1-2\alpha}}{2\alpha - 1} \leq \frac{2\alpha}{2\alpha - 1} 
%    \end{aligned}
%\end{equation} Using the same analysis as the above, we have :
%
%\begin{equation}
%    \begin{aligned}
%        \eta_L \E \bigl[\mathcal{G}^t_{\ell}]
%        & \leq \frac{GD}{L}
%        + \frac{\left(\beta C_p + C_d\right)AD}{L}\frac{2\alpha}{2\alpha-1}
%         + \frac{1}{nL}\sum_{\ell=1}^{L}\sum_{i=1}^{n}\eta_{\ell}\langle \vect{d}^t_{i,\ell}, \vect{v}^t_{i,\ell} - \vect{a}^t_{\ell} \rangle  + \frac{A^2 D^2 \beta / 2}{L}\frac{2\alpha}{2\alpha-1}\\
%        & \leq \frac{GD}{L} + \frac{\left(\beta C_p + C_d\right)AD + A^2 D^2 \beta / 2}{L}\frac{2\alpha}{2\alpha-1} + \frac{1}{nL}\sum_{\ell=1}^{L}\sum_{i=1}^{n}\frac{A}{\ell^\alpha}\langle \vect{d}^t_{i,\ell}, \vect{v}^t_{i,\ell} - \vect{a}^t_{\ell} \rangle
%    \end{aligned}
%\end{equation}
%Thus, 
%\begin{equation}
%    \begin{aligned}
%        \E \bigl[\mathcal{G}^t_{\ell}]
%        & \leq \frac{GDA^{-1}}{L^{1-\alpha}} + \frac{\left(\beta C_p + C_d\right)D + A D^2 \beta / 2}{L^{1-\alpha}}\frac{2\alpha}{2\alpha-1} + \frac{1}{nL^{1-\alpha}}\sum_{\ell=1}^{L}\sum_{i=1}^{n}\frac{1}{\ell^{\alpha}}\langle \vect{d}^t_{i,\ell}, \vect{v}^t_{i,\ell} - \vect{a}^t_{\ell} \rangle
%    \end{aligned}
%\end{equation}
%As in~(\ref{eq:case1_finalbound}), we take the mean on T and choose an oracle with regret $O \left(\sqrt{T}\right)$, the last term of the above equation is bounded by $O \left(\frac{(1-\alpha)^{-1}}{\sqrt{T}} \right)$,
%\begin{equation}
%\label{eq:case2_finalbound}
%    \begin{aligned}
%        \frac{1}{T}\sum_{t=1}^{T} \E \bigl[\mathcal{G}^t_{\ell}] 
%        & \leq & \leq \frac{GDA^{-1}}{L^{1-\alpha}} + \frac{\left(\beta C_p + C_d\right)D + A D^2 \beta / 2}{L^{1-\alpha}}\frac{2\alpha}{2\alpha-1} + O \left(\frac{(1-\alpha)^{-1}}{\sqrt{T}} \right)
%    \end{aligned}
%\end{equation} 
%\newline
%Recall that,
%\begin{align*}
%    \E \left[\mathcal{G}^t_{\ell}\right] = \E{\left[\max_{a \in \mathcal{K}}\langle \nabla F_t (\overline{\vect{x}}^t), \overline{\vect{x}}^t - a \rangle \right]}
%\end{align*}
%The theorem followed by applying lemma \ref{lemma:final_step} and setting $L=T$ in~(\ref{eq:case1_finalbound}) and~(\ref{eq:case2_finalbound}).
%\end{comment}
%\end{proof}


%%%%%%%%%%%%%%%%%%%

\subsection{An Algorithm with Stochastic Gradient Estimates}

\setcounter{lemma}{3}
\begin{lemma}[Lemma 3, \cite{zhang20_quantized:2020}]
\label{lmm:red_var}
Let $\{\vect{d}_{\ell}\}_{\ell \geq 1}$ be a sequence of points in $\mathbb{R}^n$ such that $\|\vect{d}_{\ell} - \vect{d}_{\ell-1}\| \leq \dfrac{B}{(\ell +3)^{\alpha}}$ for all $\ell \geq 1$ with fixed constant $B \geq 0$, $\alpha \in (0,1]$. Let $\{ \widetilde{\vect{d}}_{\ell} \}$ be a sequence of random variables such that $\mathbb{E} [\widetilde{\vect{d}}_{\ell}|\mathcal{H}_{\ell-1}] = \vect{d}_{\ell}$ and $\mathbb{E} \left[ \bigl \| \widetilde{\vect{d}}_{\ell} - \vect{d}_{\ell} \bigr \|^2 | \mathcal{H}_{\ell-1} \right] \leq \sigma^2$ for every $\ell \geq 1$, where $\mathcal{H}_{\ell - 1}$ is the history up to $\ell-1$. Let $\{ \widetilde{\vect{a}}_{\ell}\}_{\ell \geq 0}$ be a sequence of random variables defined recursively as  
\begin{linenomath}
    \[ \widetilde{\vect{a}}_{\ell} = (1 - \rho_{\ell}) \widetilde{\vect{a}}_{\ell-1} + \rho_{\ell} \widetilde{\vect{d}}_{\ell} \] 
\end{linenomath}
%
for $\ell \geq 1$ where  $\rho_{\ell} = \dfrac{2}{(\ell+3)^{2\alpha/3}}$ and 
$\widetilde{\vect{a}}_{0}$ is fixed. Then we have 

\begin{equation*}
        \E{\|\vect{d}_{\ell} - \Tilde{\vect{a}}_{\ell} \|^2} \leq \frac{Q}{(\ell + 4)^{2\alpha/3}}
\end{equation*}
where $Q = \max\{4^{2\alpha/3}\|\Tilde{\vect{a}}_0 - \vect{d}_0\|^2, 4\sigma^2 + 2B^2\}$
\end{lemma}

\setcounter{lemma}{4}
\begin{lemma}
Given the assumptions of Theorem~\ref{thm:stoc:version2}, for every $1 \leq t \leq T$, $1 \leq i \leq n$ and $1 \leq \ell \leq L$, it holds that 
\begin{equation*}
    \E{\|\Tilde{\vect{d}}^t_{i,\ell} - \vect{d}^t_{i,\ell} \|^2} \leq 12\left(\Tilde{\beta}^2 + \beta^2\right) \left(2C_p + AD \right)^2 + \sigma^2
\end{equation*}
\end{lemma}
\begin{proof}
    Fix an arbitrary time $t$. For any $1 \leq i \leq n$, we have 

\begin{align}	\label{eq:claim-d-1}
\E & \left[  \| \widetilde{\vect{d}}^t_{i,\ell + 1} - \vect{d}^t_{i,\ell+1} \|^2 \right] 	\notag \\
%
&= \E \biggl[ \left \| \widetilde{\nabla} f^{t}_{i}(\vect{x}^{t}_{i,\ell+1}) - \widetilde{\nabla} f^{t}_{i}(\vect{x}^{t}_{i,\ell}) - ( \nabla f^{t}_{i}(\vect{x}^{t}_{i,\ell+1}) - \nabla f^{t}_{i}(\vect{x}^{t}_{i,\ell}) )+   (\widetilde{\vect{d}}^t_{i,\ell} - \vect{d}^t_{i,\ell}) \right \|^2 \biggr] 	\notag \\
%
&= \E \biggl[ \left \| \widetilde{\nabla} f^{t}_{i}(\vect{x}^{t}_{i,\ell+1}) - \widetilde{\nabla} f^{t}_{i}(\vect{x}^{t}_{i,\ell}) - ( \nabla f^{t}_{i}(\vect{x}^{t}_{i,\ell+1})  - \nabla f^{t}_{i}(\vect{x}^{t}_{i,\ell}) ) \right \|^{2} +   \left  \| \widetilde{\vect{d}}^t_{i,\ell} - \vect{d}^t_{i,\ell} \right \|^2 \biggr] 	\notag \\
%
&\leq \E \biggl[ 4 \bigl( \| \widetilde{\nabla} f^{t}_{i}(\vect{x}^{t}_{i,\ell+1}) - \widetilde{\nabla} f^{t}_{i}(\vect{x}^{t}_{i,\ell}) \|^{2}
	+ \| \nabla f^{t}_{i}(\vect{x}^{t}_{i,\ell+1})  - \nabla f^{t}_{i}(\vect{x}^{t}_{i,\ell}) \|^{2} \bigr)
	+ \left  \| \widetilde{\vect{d}}^t_{i,\ell} - \vect{d}^t_{i,\ell} \right \|^2 \biggr]	\notag \\
%
&\leq \E \biggl [4 (\widetilde{\beta}^{2} + \beta^{2}) \| \vect{x}^{t}_{i,\ell+1} - \vect{x}^{t}_{i,\ell}\|^{2}
	+  \left  \| \widetilde{\vect{d}}^t_{i,\ell} - \vect{d}^t_{i,\ell} \right \|^2 \biggr]
%
\end{align}

The second equality holds since 
$\E \bigl[ \widetilde{\nabla} f^{t}_{i}(\vect{x}^{t}_{i,\ell+1}) - \widetilde{\nabla} f^{t}_{i}(\vect{x}^{t}_{i,\ell}) - ( \nabla f^{t}_{i}(\vect{x}^{t}_{i,\ell+1})  - \nabla f^{t}_{i}(\vect{x}^{t}_{i,\ell}) ) \bigr ] = 0$. 
The first inequality follows the fact that $\|\vect{a} + \vect{b}\|^{2} \leq 4 ( \| \vect{a} \|^{2} + \| \vect{b} \|^{2}) $.
The last inequality is due to the $\beta$-Lipschitz and $\widetilde{\nabla}$-Lipschitz of $\nabla f^{t}_{i}$ 
and $ \widetilde{\nabla} f^{t}_{i}$, respectively.

%
Moreover, 

\begin{align}	
\| \vect{x}^{t}_{i,\ell+1} - \vect{x}^{t}_{i,\ell}\| 
&\leq \| \vect{x}^{t}_{i,\ell+1} - \overline{\vect{x}}^{t}_{\ell+1} \| +  \| \overline{\vect{x}}^{t}_{\ell+1} - \overline{\vect{x}}^{t}_{\ell} \|
	+ \| \overline{\vect{x}}^{t}_{\ell} - \vect{x}^{t}_{i,\ell} \|	\notag \\
&\leq \frac{2C_{p}}{\ell} 
	+ \| \overline{\vect{x}}^{t}_{\ell+1} - \overline{\vect{x}}^{t}_{\ell} \|  \tag{by Lemma~\ref{lem:convergence}} \\
&= \frac{2C_{p}}{\ell} 
	+ \eta_{\ell} \biggl \| \frac{1}{n} \sum_{j=1}^{n} \vect{v}^{t}_{j,\ell} - \overline{\vect{x}}^{t}_{\ell} \biggr \|
 		\tag{by Lemma~\ref{lmm:avg}} \\
&\leq \frac{2C_{p}}{\ell} + \eta_{\ell} D \\
%
&\leq \frac{2C_{p} + AD}{\ell^{3/4}}	\label{eq:claim-d-2}
\end{align}

where in the last inequality, $ \bigl \| \frac{1}{n} \sum_{j=1}^{n} \vect{v}^{t}_{j,\ell} - \overline{\vect{x}}^{t}_{\ell} \bigr \| \leq D$ 
for every $t,\ell$ since both $\frac{1}{n} \sum_{j=1}^{n} \vect{v}^{t}_{j,\ell}$ and $\overline{\vect{x}}^{t}_{\ell}$ are in $\mathcal{K}$. 
Therefore, combining (\ref{eq:claim-d-1}) and (\ref{eq:claim-d-2}), we get 

\begin{align}	\label{eq:claim-d-rec}
\E \left[  \| \widetilde{\vect{d}}^t_{i,\ell + 1} - \vect{d}^t_{i,\ell+1} \|^2 \right] 
\leq 4 (\widetilde{\beta}^{2} + \beta^{2}) \frac{(2C_{p} + AD)^{2}}{\ell^{3/2}}
	+  \left  \| \widetilde{\vect{d}}^t_{i,\ell} - \vect{d}^t_{i,\ell} \right \|^2 
\end{align}

Applying (\ref{eq:claim-d-rec}) recursively on $\ell$, we deduce that

\begin{align*}
\E \left[  \| \widetilde{\vect{d}}^t_{i,\ell + 1} - \vect{d}^t_{i,\ell+1} \|^2 \right] 
&\leq 4 (\widetilde{\beta}^{2} + \beta^{2})(2C_{p} + AD)^{2} \sum_{l=1}^{\ell} \frac{1}{l^{3/2}}
	+  \left  \| \widetilde{\vect{d}}^t_{i,1} - \vect{d}^t_{i,1} \right \|^2 \\
&\leq 12 (\widetilde{\beta}^{2} + \beta^{2})(2C_{p} + AD)^{2}
	+  \left  \| \widetilde{\vect{d}}^t_{i,1} - \vect{d}^t_{i,1} \right \|^2
\end{align*}

since $ \sum_{l=1}^{\ell} \frac{1}{l^{3/2}}  \leq 3$.
Besides, for any $1 \leq i \leq n$

\begin{align*}
\E \left[  \| \widetilde{\vect{d}}^t_{i,1} - \vect{d}^t_{i,1} \|^2 \right] 
&= \E \biggl[  \biggl \|  \sum_{j} W_{ij} (\widetilde{\vect{g}}^t_{j,1} - \vect{g}^t_{j,1}) \biggr \|^2 \biggr] 
\leq \sum_{j} W_{ij} \E \biggl[  \bigl \|  \widetilde{\vect{g}}^t_{j,1} - \vect{g}^t_{j,1} \bigr \|^2 \biggr] \\
%
&= \sum_{j} W_{ij} \E \biggl[  \bigl \|  \widetilde{\nabla} f^{t}_{j}(\vect{x}^{t}_{j,1}) -  \nabla f^{t}_{j}(\vect{x}^{t}_{j,1}) \bigr \|^2 \biggr] 
\leq \sigma^{2}
\end{align*}

since $\sum_{j} W_{ij} = 1$. Hence, 

\[
\E \left[  \| \widetilde{\vect{d}}^t_{i,\ell + 1} - \vect{d}^t_{i,\ell+1} \|^2 \right] 
\leq  12(\widetilde{\beta}^{2} + \beta^{2})(2C_{p} + AD)^{2} + \sigma^{2}. 
\]

\end{proof}

\begin{claim}
It holds that, 
\begin{equation*}
    \| \vect{d}^t_{i,\ell+1} - \vect{d}^t_{i, \ell} \| \leq \frac{B}{(\ell+3)^{\alpha}} 
\end{equation*}
where $B = 4C_{d} + 2\beta \left[ 2C_{p} + AD \right]$
\end{claim}
\begin{proof}
    \begin{align}
\norm{\overline{\vect{x}}_{\ell}^{t} - \overline{\vect{x}}_{\ell-1}^{t}} &= \eta_{\ell} \norm{ \left[ \frac{1}{n} \left( \sum_{j=1}^{n} \vect{v}_{j,\ell-1}^{t} \right) \right] - \overline{\vect{x}}_{\ell-1}^{t}} \tag{By Lemma~\ref{lmm:avg}} \nonumber\\
%
%
&\leq \eta_{\ell} D \tag{$\frac{1}{n} \sum_{j=1}^{n} \overline{\vect{v}}_{j,\ell-1}^{t} \in \mathcal{K}$,  $\overline{\vect{x}}_{\ell-1} \in \mathcal{K}$ and $D = \sup_{x,y \in \mathcal{K}^2} \norm{x-y}$ }\nonumber\\
%
%
&= \frac{AD}{\ell}\tag{Definition of $\eta_{\ell} = \frac{A}{\ell}$} \\
%%%
\norm{\overline{\vect{x}}_{\ell}^{t} - \overline{\vect{x}}_{\ell-1}^{t}} &\leq \frac{AD}{\ell} \label{eq:distance_xbars}
\end{align}
\begin{align}
\norm{\vect{x}_{j,\ell}^{t} - \vect{x}_{j,\ell-1}^{t}} &\leq \norm{ \vect{x}_{j,\ell}^t-\overline{\vect{x}}_{\ell}^{t}} + \norm{\overline{\vect{x}}_{\ell}^{t}-\overline{\vect{x}}_{\ell-1}^{t}} + \norm{\overline{\vect{x}}_{\ell}^{t}-\vect{x}_{j,\ell-1}^{t}} \tag{Triangle inequality}\\
%
%
&\leq \frac{C_p}{\ell}  + \norm{\overline{\vect{x}}_{\ell}^{t} - \overline{\vect{x}}_{\ell-1}^{t}} + \frac{C_p}{\ell-1} \tag{By Lemma~\ref{lmm:avg}} \\
%
%
&\leq \frac{C_p}{\ell} + \frac{C_p}{\ell-1} + \frac{AD}{\ell} \tag{By equation~\ref{eq:distance_xbars}}\\
%%%
\norm{\vect{x}_{j,\ell}^{t} - \vect{x}_{j,\ell-1}^{t}} &\leq \frac{C_p}{\ell} + \frac{C_p}{\ell-1} + \frac{AD}{\ell} \label{eq:xjl-xjl-1}
\end{align}
\begin{align}
\norm{\vect{d}_{i,\ell}^{t} - \vect{d}_{i,\ell-1}^{t}} &\leq \norm{\vect{d}_{i,\ell}^{t} - \nabla F_{\ell}^{t}} + \norm{\nabla F_{\ell}^{t} - \nabla F_{\ell-1}^{t}} +\norm{\nabla F_{\ell-1}^{t} - \vect{d}_{i,\ell-1}^{t}} \tag{Triangle inequality}\\
%
%
&\leq \frac{C_{d}}{\ell} + \norm{\nabla F_{\ell}^{t} - \nabla F_{\ell-1}^{t}} + \frac{C_{d}}{\ell-1} \tag{By Lemma~\ref{lem:convergence}}\\
%
%
&=  \frac{C_{d}}{\ell} + \frac{C_{d}}{\ell-1} + \frac{1}{n} \sum_{j=1}^{n} \norm{\nabla f_{j}^{t}(\vect{x}_{j,\ell}^{t}) - \nabla f_{j}^{t}(\vect{x}_{j,\ell-1}^{t})} \tag{Definition of $\nabla F_{\ell}^{t}$}\\
%
%
&\leq \frac{C_{d}}{\ell} + \frac{C_{d}}{\ell-1} + \frac{\beta}{n} \sum_{j=1}^{n} \norm{\vect{x}_{j,\ell}^{t} - \vect{x}_{j,\ell-1}^{t}} \tag{$f_{j}^{t}$ is $\beta$-smooth}\\
%
%
&\leq \frac{C_{d}}{\ell} + \frac{C_{d}}{\ell-1} + \beta \left[ \frac{C_p}{\ell} + \frac{C_p}{\ell-1} + \frac{AD}{\ell}\right]  \tag{By equation~\ref{eq:xjl-xjl-1}}\\
%
%
&\leq \frac{2C_{d}}{\ell+3} + \frac{2C_{d}}{\ell+3} + \beta \left[ \frac{2C_p}{\ell+3} + \frac{2C_p}{\ell+3} + \frac{2AD}{\ell+3}\right]  \tag{When $\ell \geq 7$}  \\
%
%
&= \frac{4C_{d} + 2\beta \left[ 2C_{p} + AD \right]}{\ell+3}\\
%
&\leq \frac{4C_{d} + 2\beta \left[ 2C_{p} + AD \right]}{(\ell+3)^{\alpha}}
\end{align}
\end{proof}

\begin{remark}
\label{rmk:quantized_fw}
By Lemma \ref{lmm:red_var} and Jensen's inequality, we can deduce the following inequality
\begin{equation*}
    \E{\|\vect{d}^t_{i,\ell} - \Tilde{\vect{a}}^t_{i,\ell} \|} \leq \sqrt{\E{\|\vect{d}^t_{i,\ell} - \Tilde{\vect{a}}^t_{i,\ell} \|^{2}}} \leq \frac{Q^{1/2}}{(\ell + 4)^{1/4}}
\end{equation*}
\end{remark}

\begin{theorem}
\label{thm:stoc:version2}
Let $\mathcal{K}$ be a convex set with diameter $D$. Assume that for every $1 \leq t \leq T$, 
\begin{enumerate}
	\item functions $f^{t}_{i}$ are $\beta$-smooth, i.e. $\nabla f^{t}_{i}$ is $\beta$-Lipschitz,  (so $F^{t}$ is $\beta$-smooth);
	\item $\| \nabla f^{t}_{i}\| \leq G$ (so $\| \nabla F^{t}\| \leq G$);
	\item the gradient estimates are unbiased with bounded variance $\sigma^{2}$, i.e., 
		$\E [\widetilde{\nabla} f^{t}_{i}(\vect{x}^{t}_{i,\ell})] = \nabla f^{t}_{i}(\vect{x}^{t}_{i,\ell})$
		and $\bigl \| \widetilde{\nabla} f^{t}_{i}(\vect{x}^{t}_{i,\ell})] - \nabla f^{t}_{i}(\vect{x}^{t}_{i,\ell}) \bigr \| \leq \sigma$
		for every $1 \leq i \leq n$ and $1 \leq \ell \leq L$;
	\item the gradient estimates are $\widetilde{\beta}$-Lipschitz.
\end{enumerate}
Then, choosing the step-sizes $\eta_\ell = \min \{1, \frac{A}{\ell^{3/4}}\}$ where $A \in \mathbb{R}_+$. For all $1 \leq i \leq n$, 
\begin{align*}
    \max_{\vect{o} \in \mathcal{K}} \E \left[ \frac{1}{T} \sum_{t=1}^T \E_{\vect{x}_i^t}\left[ \langle \nabla F_t\left( \vect{x}_i^t \right), \vect{x}_i^t - \vect{o} \rangle \right] \right]
    &\leq \frac{DG + 2ADQ^{1/2}}{L^{1/4}} + \frac{2AD^2\beta}{L^{3/4}} \\
    &+\left[ \left(\beta D + G\right) + \left(\beta C_p + C_d\right)D \right]  \frac{\log L }{L} + O \left(\mathcal{R}^T \right)
\end{align*}
Choosing $L=T$ and oracles as gradient descent or follow-the-perturbed-leader with regret $\mathcal{R}^T =
O\left(T^{-1/2}\right)$, we obtain a convergence rate of $O\left( T^{-1/4} \right).$

\end{theorem}


\begin{proof}
By equation (\ref{tk:exp_gap}) in the proof of Theorem \ref{thm:gap}, we have:
\begin{align*}
    \E_{\overline{x}^t}\left[ \mathcal{G}^t \right] 
    &\leq \frac{DG}{L^{1/4}A} + \frac{(\beta C_p + C_d) D \log L}{L} + \frac{2\beta AD^2}{L^{3/4}} + \frac{1}{nL}\sum_{\ell=1}^{L} \sum_{i=1}^n \langle \vect{d}_{i,\ell}^t, \vect{v}_{i,\ell}^t - \vect{o}_{\ell}^t \rangle\\
    %
    &\leq \frac{DG}{L^{1/4}A} + \frac{(\beta C_p + C_d) D \log L}{L} + \frac{2\beta AD^2}{L^{3/4}} + \frac{1}{nL}\sum_{\ell=1}^{L} \sum_{i=1}^n \langle \vect{d}_{i,\ell}^t -\widetilde{\vect{a}}_{i,\ell}^t,\vect{v}_{i,\ell}^t - \vect{o}_{\ell}^t \rangle \\
     & \quad + \frac{1}{nL}\sum_{\ell=1}^{L} \sum_{i=1}^n \langle \widetilde{\vect{a}}_{i,\ell}^t,\vect{v}_{i,\ell}^t - \vect{o}_{\ell}^t \rangle\\
    %
    &\leq \frac{DG}{L^{1/4}A} + \frac{(\beta C_p + C_d) D \log L}{L} + \frac{2\beta AD^2}{L^{3/4}} + \frac{1}{nL}\sum_{\ell=1}^{L} \sum_{i=1}^n \|\vect{d}_{i,\ell}^t -\widetilde{\vect{a}}_{i,\ell}^t\| \| \vect{v}_{i,\ell}^t - \vect{o}_{\ell}^t \| \\
     & \quad + \frac{1}{nL}\sum_{\ell=1}^{L} \sum_{i=1}^n \langle \widetilde{\vect{a}}_{i,\ell}^t,\vect{v}_{i,\ell}^t - \vect{o}_{\ell}^t \rangle
     \tag{Cauchy-Schwarz}\\
    %
    &\leq \frac{DG}{L^{1/4}A} + \frac{(\beta C_p + C_d) D \log L}{L} + \frac{2\beta AD^2}{L^{3/4}} + \frac{D}{nL}\sum_{\ell=1}^{L} \sum_{i=1}^n \|\vect{d}_{i,\ell}^t -\widetilde{\vect{a}}_{i,\ell}^t\| \\
     & \quad + \frac{1}{nL}\sum_{\ell=1}^{L} \sum_{i=1}^n \langle \widetilde{\vect{a}}_{i,\ell}^t,\vect{v}_{i,\ell}^t - \vect{o}_{\ell}^t \rangle
     \tag{$\vect{v}_{i,\ell}^t, \vect{o}_{\ell}^t \in \mathcal{K}^2 \Rightarrow \| \vect{v}_{i,\ell}^t -  \vect{o}_{\ell}^t \| \leq D$}
\end{align*}

\begin{align*}
    \E\left[ \frac{1}{T} \sum_{t=1}^T \E_{\overline{x}^t} \left[ \mathcal{G}^t \right] \right]
    &\leq \frac{DG}{L^{1/4}A} + \frac{(\beta C_p + C_d) D \log L}{L} + \frac{2\beta AD^2}{L^{3/4}} + \frac{D}{nLT}\sum_{\ell=1}^{L} \sum_{i=1}^n \sum_{t=1}^T  \E \left[ \|\vect{d}_{i,\ell}^t -\widetilde{\vect{a}}_{i,\ell}^t\| \right] \\
     & \quad + \E \left[ \frac{1}{nLT}\sum_{\ell=1}^{L} \sum_{i=1}^n \sum_{t=1}^T \langle \widetilde{\vect{a}}_{i,\ell}^t,\vect{v}_{i,\ell}^t - \vect{o}_{\ell}^t \rangle \right] \\
     %
     &\leq \frac{DG}{L^{1/4}A} + \frac{(\beta C_p + C_d) D \log L}{L} + \frac{2\beta AD^2}{L^{3/4}} + \frac{Q^{1/2}D}{L}\sum_{\ell=1}^{L} \frac{1}{(\ell+4)^{1/4}} \\
     & \quad + \E \left[ \frac{1}{nLT}\sum_{\ell=1}^{L} \sum_{i=1}^n \sum_{t=1}^T \langle \widetilde{\vect{a}}_{i,\ell}^t,\vect{v}_{i,\ell}^t - \vect{o}_{\ell}^t \rangle \right]
     \tag{By remark \ref{rmk:quantized_fw}}\\
     %
     &\leq \frac{DG}{L^{1/4}A} + \frac{(\beta C_p + C_d) D \log L}{L} + \frac{2\beta AD^2}{L^{3/4}} + \frac{2Q^{1/2}D}{L^{1/4}}  \\
     & \quad + \E \left[ \frac{1}{nLT}\sum_{\ell=1}^{L} \sum_{i=1}^n \sum_{t=1}^T \langle \widetilde{\vect{a}}_{i,\ell}^t,\vect{v}_{i,\ell}^t - \vect{o}_{\ell}^t \rangle \right]
     \tag{$\sum_{\ell=1}^L \frac{1}{(\ell + 4)^{1/4}}\leq 2 L^{3/4}$}\\
     %
     &\leq \frac{DG}{L^{1/4}A} + \frac{(\beta C_p + C_d) D \log L}{L} + \frac{2\beta AD^2}{L^{3/4}} + \frac{2Q^{1/2}D}{L^{1/4}} + O \left(\mathcal{R}^T\right) \\
     \tag{$\vect{v}^t_{i,\ell}$ are chosen by the online oracles with regret $\mathcal{R}^T$}
\end{align*}
%
Recall that, 
\begin{align*}
    \E_{\overline{\vect{x}}^t}{ \left[ \mathcal{G}^t \right]}
    &= \E_{\overline{\vect{x}}^t}{\left[ \max_{\vect{o} \in \mathcal{K}} \langle \nabla F_t\left( \overline{\vect{x}}^t \right), \overline{\vect{x}}^t - \vect{o} \rangle \right]} 
\end{align*}
Therefore by lemma \ref{lemma:final_step}, 
\begin{align*}
    \E \left[ \frac{1}{T} \sum_{t=1}^T \E_{\vect{x}_i^t}\left[ \max_{\vect{o} \in \mathcal{K}} \langle \nabla F_t\left( \vect{x}_i^t \right), \vect{x}_i^t - \vect{o} \rangle \right] \right]
    &\leq \E \left[ \frac{1}{T} \sum_{t=1}^T  \E_{\overline{\vect{x}}^t}{\left[ \max_{\vect{o} \in \mathcal{K}} \langle \nabla F_t\left( \overline{\vect{x}}^t \right), \overline{\vect{x}}^t - \vect{o} \rangle \right]} \right] \\
    & \quad + \frac{(\beta D + G)C_p \log L}{L} \\
    %
    &\leq \frac{DG + 2ADQ^{1/2}}{L^{1/4}} + \frac{2AD^2\beta}{L^{3/4}} \\
    &+\left[ \left(\beta D + G\right) + \left(\beta C_p + C_d\right)D \right]  \frac{\log L }{L} + O \left(\mathcal{R}^T \right) 
\end{align*}
%
Since $\max$ is a convex function, the theorem follows by applying Jensen's inequality on the left-hand side of the above equation.
\end{proof}

% \include{chapters/appendix_prediction}
\end{document}
